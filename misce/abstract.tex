\addcontentsline{toc}{chapter}{\textbf{Abstract}}
\begin{center}
    {\Large\bfseries\sffamily Abstract}
\end{center}
One of the major challenges in unifying quantum mechanics with general relativity lies in reconciling the principle of general covariance, or background independence, with the quantum framework. In such a unified theory, the Hamiltonian is constrained to vanish identically ($\oper{H} \equiv 0$). This constraint presents the ``problem of time" -  implying that the quantum states should not evolve with respect to an external, classical time coordinate.  On the other hand, just as in local Quantum Field theories where a superselection rule applies to charges, there has been speculation about the existence of a similar super-selection rule for energy. This is because energy couples to long-range gravitational fields in the same way that charge couples to the Coulomb field. Consequently, one would expect to observe no explicit time dependence in any quantities that can be experimentally measured.

Two main approaches address these problems. The first relies on semiclassical approximations which inherently carry over the classical notion of time. The second approach, known as Relational Quantum Mechanics, takes a purely quantum-mechanical perspective. This framework, pioneered by Page and Wootters~\cite{page1983evolution}, focuses on the relationship between a system of interest and a reference system, often called a ``clock" or ``environment." By conditioning the global quantum state on the state of the environment, one can extract the dynamics of the system relative to its reference ``clock". Recent work by~\cite{Gemsheim:2023izg} has shown that this relational approach can be extended to handle complex interacting quantum systems.

This thesis investigates the classical limit of the quantum relational approach. Specifically, we focus on analyzing the JC-Hamiltonian within the relational paradigm and identify the key conditions under which the environment can be considered classical. Finally, we compare the results obtained through this method with those predicted by established semiclassical theories. Our analysis demonstrates that, under appropriate limits, the classical limit of relational dynamics recovers the well-known results expected from semi-classical approaches. This outcome strengthens the connection between these two frameworks, opening exciting new avenues for future research.
\newpage