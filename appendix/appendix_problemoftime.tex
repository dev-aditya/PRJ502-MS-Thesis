\chapter{The Problem of Time\label{appen:problemoftime}}

Action principles are widely used to express the laws of physics, including those of
general relativity.  Symmetry transformations are changes in the coordinates or
variables that leave the action invariant. It is well known that continuous symmetries
generate conservation laws (Noether’s Theorem). Conservation laws are of fundamental importance in physics, and so it is valuable to investigate the symmetries of the action.

It is useful to distinguish between two types of symmetries: \emph{dynamical symmetries}
corresponding to some inherent property of matter or spacetime evolution (e.g., the
Lagrangian being independent of a coordinate, leading to a conserved conjugate momentum) 
and \emph{non-dynamical symmetries} arising because of the way in which we formulate
the action (e.g., the gauge symmetries). Dynamical symmetries constrain the solutions
of the equations of motion, while non-dynamical symmetries give rise to special laws
called identities. They are distinct from conservation laws because they hold regardless of whether
or not one has extremized the action.

\section*{Parameterization-Invariance and Hamiltonian Constrain}
Consider a system with $n$ degrees of freedom - the generalized coordinates $q_i$ - with a parameter $t$ giving the evolution of the trajectory in configuration space. We will remove the superscript
on $q_i$ when it is clear from the context. Let the action of this system be:
\begin{equation}
	\label{eq:action_t}
	\mathcal{S} = \int L_s\left(q, \frac{dq}{dt}\right)  dt
\end{equation}
Now consider a new integration parameter \(\tau\), which now parameterize the trajectory and promote \(t \to t(\tau)\) i.e to  a dynamical variable.
In terms of \(\tau\) the action (\refeq{eq:action_t}) can be expressed as:
\begin{equation}
	\mathcal{S} = \int L_s\left(q, \frac{\Dot{q}}{\Dot{t}}\right)  \Dot{t} d\tau =  \int L\left(q, \Dot{q}, \Dot{t}\right)   d\tau
\end{equation}
where \(\dot{a} \equiv \dfrac{d a}{d \tau}\) and 
\(L\left(q, \Dot{q}, \Dot{t}\right) = \Dot{t}L_s\left(q, \frac{\Dot{q}}{\Dot{t}}\right)  \).
The Hamiltonian for the modified Lagrangian is then obtained by taking the Legendre
transformation w.r.t. both \(\dot{q} \text{ and } \dot{t}\)\ ~\cite{deriglazov2011reparametrization}:
\begin{equation}
	\label{eq:hamlt_L}
	\begin{gathered}
		H =  p_t \dot{t} + p_q \dot{q} - L\\
		H =  p_t \dot{t} + \dot{t} p_q (\dot{q}/\dot{t}) - \dot{t}L_s\\
		H  = \dot{t} \left(p_t + p_q q'- L_s\right)
	\end{gathered}
\end{equation}
where \(q'=\dfrac{dq}{dt}=(\dot{q}/\dot{t})\).

Let's calculate the conjugate momenta:
\begin{equation}
	\begin{gathered}
		p_q := \frac{\partial L}{\partial\dot{q}}
		= \dot{t}\frac{\partial L_s}{\partial\dot{q}}
		=\frac{\partial L_s}{\partial(\dot{q}/ \dot{t})}\\
		p_q = \frac{\partial L_s}{\partial q'}
	\end{gathered}
\end{equation}
which coincides with the momentum conjugate to $q$ defined by $L_s(q, q′)$. Hence, (\refeq{eq:hamlt_L}) get's modified as:
\begin{equation}
	\label{eq:hamil_contrain_Hs}
	H  = \dot{t} \left(p_t + H_s\right)
\end{equation}
where \(H_s = p_q q'- L_s\) i.e Hamiltonian conjugate to \(L_s\).

We have:
\begin{equation}
	\label{eq:conh_pt}
	\begin{gathered}
		p_t := \frac{\partial L}{\partial\dot{t}}\\
		p_t = L_s + \dot{t} \frac{\partial L_s(q, q')}{\partial\dot{t}}\\
		p_t = L_s + \dot{t} \left(\frac{\partial L_s(q, q')}{\partial q} \cancelto{0}{\frac{d q}{d \dot{t}}} +
		\frac{\partial L_s(q, q')}{\partial q'} \frac{d (\dot{q}/\dot{t})}{d \dot{t}}
		\right) \\
		p_t = L_s - q'p_s = - H_s
	\end{gathered}
\end{equation}
Using (\refeq{eq:conh_pt}) in (\refeq{eq:hamil_contrain_Hs}), the equation reduces to:
\begin{equation}
	\label{eq:H0}
	\boxed{H = 0}
\end{equation}
Therefore, for parameterization-invariant theory, the Hamiltonian function 
is identically zero. It is crucial to note that this \emph{derivation makes no 
assumptions about the extremality of the action or the satisfaction of the 
Euler-Lagrange equations} by the ($q$'s). Consequently, \refeq{eq:H0}
represents a non-dynamical symmetry.

Identity $H = 0$ is very different from conservation law $H = \mathrm{constant}$ arising
from a time-independent Lagrangian. The conservation law holds only for solutions of the
equations of motion; in contrast, when the action is parameterization-invariant, 
$H = 0$ holds for any trajectory. The non-dynamical symmetry, therefore, does not
constrain the motion.