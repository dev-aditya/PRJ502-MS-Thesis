\chapter{Diagonalization and Semiclassical limit of JC-model\label{appen:chap5_JC_calculations}}

Our initial task is to achieve the block-diagonalization of
\ref{eq:chap5_JCM_Hamiltonian_resonance}, followed by the diagonalization 
of these blocks. A quick check shows that the ground state of 
the atom and the vacuum state of the mode constitute eigenstates of 
the Hamiltonian, represented as
\begin{equation}
    \label{eq:appen_JCM_ground_state}
    \oper{H}\ket{-, 0} = -\frac{\hbar \omega}{2} \ket{-, 0}.
\end{equation}
This results in a $1 \times 1$ block in the decomposition. 
Furthermore, we observe that
\begin{equation}
    \label{eq:appen_JCM_excited_state}
    \begin{gathered}
        \oper{H}\ket{+, n-1} = \left( \frac{\hbar \omega}{2} + \hbar \omega (n-1) \right) 
        \ket{+, n-1} + \hbar g \sqrt{n} \ket{-, n},\\
    \oper{H}\ket{-, n} = \left( -\frac{\hbar \omega}{2} + \hbar \omega n \right) \ket{-, n} + \hbar g \sqrt{n} \ket{+, n-1}.
    \end{gathered}
\end{equation}

This implies that \( \oper{H} \) maps all elements within the subspace spanned
by \\
\( \{ \ket{+, n - 1}, \ket{-, n} \} \) back to that same subspace.
In essence, \( \oper{H} \) block-diagonalizes concerning these subspaces. 
Using~\refeq{eq:appen_JCM_excited_state}, we can derive the matrix elements of the 
corresponding block matrices, i.e., 
\begin{equation}
\label{eq:appen_JCM_block_matrix}
\begin{bmatrix}
\langle +, n - 1 \lvert \oper{H} \lvert +, n - 1 \rangle & \langle +, n - 1 \lvert \oper{H} \lvert -, n \rangle \\
\langle -, n \lvert \oper{H} \lvert +, n - 1 \rangle & \langle -, n \lvert \oper{H} 
\lvert -, n \rangle \\
\end{bmatrix}
= 
\begin{bmatrix}
0 & \hbar g \sqrt{n} \\
\hbar g \sqrt{n} & 0 \\
\end{bmatrix}
+ \hbar \omega \left( n - \frac{1}{2} \right) \oper{I}
\end{equation}
The eigenvalues and eigenvectors of this matrix can be expressed as 
\begin{equation}
    \label{eq:appen_JCM_block_eigenvalues}
    E_{n,\pm} = \pm \hbar g \sqrt{n} + \hbar \omega \left( n - \frac{1}{2} \right),  
     \ket*{\psi_{n,\pm}} := \frac{1}{\sqrt{2}} 
     \left( \ket*{+, n - 1} \pm \ket*{-, n} \right)
\end{equation}
and finally, the uniraty evolution operator can be expressed as
\begin{eqnarray}
    \begin{gathered}
        U(t) = e^{-i t H/\hbar} = e^{\frac{1}{2} i t \omega } 
        \ket*{-, 0}\bra*{-, 0} + \sum_{n=1}^{\infty} e^{-i t \omega (n - \frac{1}{2})} e^{-i t g \sqrt{n}} \ket*{\psi_{n,+}}\bra*{\psi_{n,+}} \\
     + \sum_{n=1}^{\infty} e^{-i t \omega (n - \frac{1}{2})} e^{i t g \sqrt{n}} \ket*{\psi_{n,-}}\bra*{\psi_{n,-}} \\
    = e^{\frac{1}{2} i t \omega }\ket*{-, 0}\bra*{-, 0} \\
     + \sum_{n=1}^{\infty} e^{-i t \omega (n - \frac{1}{2})} \Big[ \cos(t g \sqrt{n}) \left( \ket*{+, n - 1}\bra*{+, n - 1} + \ket*{-, n}\bra*{-, n} \right) \\
     - i \sin(t g \sqrt{n}) \left( \ket*{+, n - 1}\bra*{-, n} + \ket*{-, n}\bra*{+, n - 1} \right) \Big].
    \end{gathered}
\end{eqnarray}

\section*{Semiclassical Hamiltonain in Large Amplitude and Weak Interaction Limit}

The emergence of the semi-classical model occurs when fields are in 
coherent states. However, it's not straightforward to recover the 
semi-classical behavior by merely substituting coherent states, (as we did in \refeq{eq:chap5_JCM_Hamiltonian_semi}). 
To elaborate, we can expect this to happen under specific conditions~\cite{semiclassical_limit_JC_PRL}:
\begin{itemize}
    \item \textbf{Large Amplitude of Coherent State}: When the amplitude of the coherent state is substantial, 
    the quantum fluctuations (i.e. the influence of atom onto the field in our model!) become less significant, 
    resembling classical behavior.
    \item \textbf{Weak Atom-Mode Coupling}: If the coupling between the atom and the mode (field) is weak, 
    the quantum effects become less pronounced, favoring semi-classical behavior.
\end{itemize}
Here, we will attempt to make this idea more concrete by considering the 
limit where $g = \frac{\Omega}{|\alpha|}$, and $|\alpha| \rightarrow \infty$, where 
$\Omega$ is a constant. In the following discussion, we will argue, albeit not in a particularly rigorous manner, that we can regain a unitary evolution on the atom in this limit. Let's first recall the form of coherent states:
\begin{equation}
    \ket{\alpha} = e^{-\frac{1}{2}|\alpha|^2} \sum_{n=0}^{\infty} \frac{\alpha^n}{\sqrt{n!}} \ket{n}
\end{equation}
    
One particular property of coherent states is that the particle number is Poisson distributed. In other words, the probability to find $n$ photons in the mode is given by:
\begin{equation}
    p_n = \abs*{\bra*{n}\ket*{\alpha}}^2 = \frac{e^{-\gamma} \gamma^n}{n!}, \text{ where } \gamma = |\alpha|^2
\end{equation}
Our objective is to understand the evolution of the atom when it interacts with a coherent state. 
Therefore, we aim to compute the reduced density operator of the atom, which is given by:

\begin{equation}
\oper{\rho}_{\text{atom}}(t) = 
\text{Tr}_{\text{mode}} \left( U(t)[\oper{\rho} \otimes \ket{\alpha}\bra{\alpha}]U^\dagger(t) \right)
 = \sum_n e^{-\abs{\alpha}^2} \abs{\alpha}^{2n} \frac{V_g(n)\oper{\rho} V_g^\dagger(n)}{n!},
\end{equation}

where

\begin{equation}
    \label{eq:appen_JCM_V_g}
\begin{gathered}
    V_g (n) =e^{-it\omega\frac{1}{2}}  \cos(tg\sqrt{n + 1})\ket{+}\bra{+}\\
+ e^{it\omega\frac{1}{2}} \cos(tg\sqrt{n})\ket{-}\bra{-}\\
- ie^{-it\omega\frac{1}{2}} \sin(tg\sqrt{n + 1}) \frac{\alpha}{\sqrt{n + 1}} \ket{+}\bra{-}\\
- ie^{-t\omega\frac{1}{2}} \sin(tg\sqrt{n}) \frac{\sqrt{n}}{\alpha} \ket{-}\bra{+}.
\end{gathered}
\end{equation}
Let's discuss a few key points. Firstly, as \( x \) increases, the function 
\( \sqrt{x} \) becomes flatter (its derivative approaches zero). Secondly, 
for large values of \( |\alpha| \), the Poisson distribution is approximately 
equivalent to the normal distribution, 
\begin{equation}
f(x) = \frac{1}{\sqrt{2\pi\sigma^2}} e^{-\frac{(x-\mu)^2}{2\sigma^2}}.
\end{equation}

In this context, the approximate normal distribution has a mean of \( |\alpha|^2 \) and a 
standard deviation of \( |\alpha| \). This implies that the distribution's width increases 
at a slower rate than the mean. Consequently, for large \( |\alpha| \), it is plausible 
that the square root remains nearly constant within the region where the Poisson/normal 
distribution is significant. Considering these observations, the subsequent reasoning 
for approximation might be easier to grasp.

\begin{equation}
\oper{\rho}_{\text{atom}}(t) = \sum_{n} e^{-|\alpha|^2} \frac{|\alpha|^{2n}}{n!} V(t, n) \oper{\rho} V(t, n)^{\dagger}
\end{equation}
By approximating the weighted sum over the Poisson distribution with the integral weighted by 
the normal distribution \( f_{|\alpha|^2, |\alpha|}(x) \), having a mean (\( \mu = |\alpha|^2 \)) 
and standard deviation (\( \sigma = |\alpha| \)), we obtain

\begin{equation}
    \oper{\rho}_{\text{atom}}(t)\approx \int f_{|\alpha|^2, |\alpha|}(x) V(t, x) \oper{\rho} V(t, x)^{\dagger} dx
\end{equation}
We cut away the tails of the integral at (\( s \)) standard deviations

\begin{equation}
    \oper{\rho}_{\text{atom}}(t)\approx \int_{|\alpha|^2 - s|\alpha|}^{|\alpha|^2 + s|\alpha|} f_{|\alpha|^2, |\alpha|}(x) V(t, x) \oper{\rho} V(t, x)^{\dagger} dx
\end{equation}
Change of variables (\( x \to x|\alpha| + |\alpha|^2 \))

\begin{equation}
    \label{eq:appen_JCM_rho_atom_t}
    \begin{aligned}
        \oper{\rho}_{\text{atom}}(t)= \int_{-s}^{s} f_{0,1}(x) V(x|\alpha| + |\alpha|^2) 
    \oper{\rho} V(x|\alpha| + |\alpha|^2)^{\dagger} dx
    \end{aligned}
\end{equation}
Now we put in our assumption that \(g = \Omega / \abs*{\alpha}\). Let's assume, 
for simplicity, that \(\alpha = \abs*{\alpha}\), i.e., it's real. Then \refeq{eq:appen_JCM_V_g} becomes
\begin{align}
    V_{\Omega/|\alpha|}(x|\alpha| + |\alpha|^2) 
    &= e^{-i t \omega \frac{1}{2}}\cos \left( t\Omega \sqrt{1 + \frac{x}{\abs*{\alpha}}+ \frac{1}{\abs*{\alpha}^2}} \right) \ket{+}\bra{+} \nonumber \\
    &+ e^{i t \omega \frac{1}{2} }\cos \left( t\Omega \sqrt{1 + \frac{x}{\abs*{\alpha}}} \right) \ket{-}\bra{-} \nonumber \\
    &- i e^{-i t \omega \frac{1}{2} }\sin \left( t\Omega \sqrt{1 + \frac{x}{\abs*{\alpha}} + \frac{1}{\abs*{\alpha}^2}} \right) \frac{1}{\sqrt{1 + \frac{x}{\abs*{\alpha}} + 1}} \ket{+}\bra{-} \nonumber \\
    &- i e^{i t \omega \frac{1}{2}} \sin \left( t\Omega \sqrt{1 + \frac{x}{\abs*{\alpha}}} \right) \sqrt{1 + \frac{x}{\abs*{\alpha}}} \ket{-}\bra{+}. \nonumber
\end{align}
Since, \(-s < x < s\), with \(s\) constant, in the limit of \(|\alpha| \to \infty\),
\begin{align}
    \tilde{U}(t) &:= \lim_{\abs*{\alpha}^2 \to \infty} 
    V_{\Omega/|\alpha|}(x|\alpha| + |\alpha|^2)  \nonumber \\
    &= \cos(\Omega t) \left(e^{-i t \omega \frac{1}{2}}\ket{+}\bra{+} + 
    e^{i t \omega \frac{1}{2}}\ket{-}\bra{-} \right)  \nonumber \\
    &- i \sin(\Omega t)
     \left( e^{-i t \omega \frac{1}{2}}\ket{+}\bra{-}
    + e^{i t \omega \frac{1}{2}}\ket{-}\bra{+}\right)  \nonumber \\
\end{align}
Inserting this limit back into \refeq{eq:appen_JCM_rho_atom_t}, we obtain
\begin{align}
    \oper{\rho}_{\text{atom}}(t) &\approx \int_{-s}^{s} f_{0,1}(x) \tilde{U}(t) \oper{\rho} \tilde{U}(t)^{\dagger} dx \nonumber\\
        &\approx \tilde{U}(t) \oper{\rho} \tilde{U}(t)^{\dagger}.
\end{align}

Hence in the limit of large \(\alpha\) and weak interaction, we get the 
semiclassical limit. 