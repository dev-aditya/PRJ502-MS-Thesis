\chapter{Diagonalization of JC-model\label{appen:chap5_JC_calculations}}

Our initial task is to achieve the block-diagonalization of
\ref{eq:chap5_JCM_Hamiltonian_resonance}, followed by the diagonalization 
of these blocks. A quick check shows that the ground state of 
the atom and the vacuum state of the mode constitute eigenstates of 
the Hamiltonian, represented as
\begin{equation}
    \label{eq:appen_JCM_ground_state}
    \oper{H}\ket{-, 0} = -\frac{\hbar \omega}{2} \ket{-, 0}.
\end{equation}
This results in a $1 \times 1$ block in the decomposition. 
Furthermore, we observe that
\begin{equation}
    \label{eq:appen_JCM_excited_state}
    \begin{gathered}
        \oper{H}\ket{+, n-1} = \left( \frac{\hbar \omega}{2} + \hbar \omega (n-1) \right) 
        \ket{+, n-1} + \hbar g \sqrt{n} \ket{-, n},\\
    \oper{H}\ket{-, n} = \left( -\frac{\hbar \omega}{2} + \hbar \omega n \right) \ket{-, n} + \hbar g \sqrt{n} \ket{+, n-1}.
    \end{gathered}
\end{equation}

This implies that \( \oper{H} \) maps all elements within the subspace spanned
by \\
\( \{ \ket{+, n - 1}, \ket{-, n} \} \) back to that same subspace.
In essence, \( \oper{H} \) block-diagonalizes concerning these subspaces. 
Using~\refeq{eq:appen_JCM_excited_state}, we can derive the matrix elements of the 
corresponding block matrices, i.e., 
\begin{equation}
\label{eq:appen_JCM_block_matrix}
\begin{bmatrix}
\langle +, n - 1 \lvert \oper{H} \lvert +, n - 1 \rangle & \langle +, n - 1 \lvert \oper{H} \lvert -, n \rangle \\
\langle -, n \lvert \oper{H} \lvert +, n - 1 \rangle & \langle -, n \lvert \oper{H} 
\lvert -, n \rangle \\
\end{bmatrix}
= 
\begin{bmatrix}
0 & \hbar g \sqrt{n} \\
\hbar g \sqrt{n} & 0 \\
\end{bmatrix}
+ \hbar \omega \left( n - \frac{1}{2} \right) \oper{I}
\end{equation}
The eigenvalues and eigenvectors of this matrix can be expressed as 
\begin{equation}
    \label{eq:appen_JCM_block_eigenvalues}
    E_{n,\pm} = \pm \hbar g \sqrt{n} + \hbar \omega \left( n - \frac{1}{2} \right),  
     \ket*{\psi_{n,\pm}} := \frac{1}{\sqrt{2}} 
     \left( \ket*{+, n - 1} \pm \ket*{-, n} \right)
\end{equation}
and finally, the uniraty evolution operator can be expressed as
\begin{align}
    U(t) = e^{-i t H/\hbar} 
    &= e^{\frac{1}{2} i t \omega } \ket*{-, 0}\bra*{-, 0} + \sum_{n=1}^{\infty} e^{-i t \omega (n - \frac{1}{2})} e^{-i t g \sqrt{n}} \ket*{\psi_{n,+}}\bra*{\psi_{n,+}} \nonumber \\
     &+ \sum_{n=1}^{\infty} e^{-i t \omega (n - \frac{1}{2})} e^{i t g \sqrt{n}} \ket*{\psi_{n,-}}\bra*{\psi_{n,-}} \nonumber \\
    &= e^{\frac{1}{2} i t \omega }\ket*{-, 0}\bra*{-, 0} \nonumber \\
     &+ \sum_{n=1}^{\infty} e^{-i t \omega (n - \frac{1}{2})} \Big[ \cos(t g \sqrt{n}) \left( \ket*{+, n - 1}\bra*{+, n - 1} + \ket*{-, n}\bra*{-, n} \right) \nonumber \\
     &- i \sin(t g \sqrt{n}) \left( \ket*{+, n - 1}\bra*{-, n} + \ket*{-, n}\bra*{+, n - 1} \right) \Big].
\end{align}
