\chapter{Super-Selection Rule\label{appen:superselection}}


In quantum theory, physically measurable quantities of a microscopic system are represented by self-adjoint operators. However, not all of the self-adjoint operators correspond to measurable quantities. The super-selection rule is a criterion to distinguish measurable self-adjoint operator from the un-measurable ones, i.e., any measurable quantity must obey the superselection rules. By contraposition, any quantity which does not obey the superselection rules cannot be measured. It should be noted that
such a statement implies that the set of (physically realizable) observables is strictly
smaller than the set of all self-adjoint operators on Hilbert space. The notion of superselection rule (henceforth abbreviated SSR) was first introduced
in 1952 by Wick, Wightman, and Wigner~\cite{Wick:1952nb} in connection with the problem of consistently assigning intrinsic parity to elementary particles.


In the context of quantum field theory, the electric current \( J^\mu  = \overline{\psi}\gamma^\mu \psi\) is defined in terms of the Dirac spinor field operator \(\psi\) for electrons. The electric current $J^\mu$ is self-adjoint and measurable. However, the operators
\begin{equation}
	\frac{1}{2}\left(\psi + \psid \right) \qquad \frac{1}{2i}\left(\psi - \psid \right)
\end{equation}
are also self-adjoint but they are not measurable even via indirect methods.




A SSR is stated as follows: There is an operator $\oper{J}$, which we call the
superselection charge. If a self-adjoint operator $\oper{A}$ represents a measurable quantity, it must
satisfy the commutativity~\cite{TanimuraSuperSelection}

\begin{equation}
	\label{eq:superselection_law}
	\comm{\oper{J}}{\oper{A}} = 0
\end{equation}

This is a SSR, which is a necessary condition for the measurability of $\oper{A}$.
The SSR can be compared with a conservation law. The conservation of
$\oper{J}$ is formulated as

\begin{equation}
	\label{eq:conservation_law}
	\comm{\oper{J}}{\oper{H}} = 0
\end{equation}
where $\oper{H}$ is the Hamiltonian $\oper{H}$ of the system. The conservation law ~(\refeq{eq:conservation_law}) requires that
$\oper{J}$ commutes with the Hamiltonian $\oper{H}$ while the superselection rule ~(\refeq{eq:superselection_law}) requires that $\oper{J}$ commutes with all of the measurable quantities. Thus, the superselection rule is a stronger requirement for $\oper{J}$ than the conservation law. It can be said that the superselection rule is an extreme form of conservation laws.

Notice that the SSR~(Eq~\refeq{eq:superselection_law}) implies that, for all  physically measurable observables $\oper{A}$ and  any eigenvector \(\ket{\psi _i}\) of $\oper{J}$ with charge $q_i$\footnote{we assume $\oper{J}$ is non degenerate}.
\begin{equation}
	\label{eq:sSSR_orthoganility}
	\begin{gathered}
		\bra{\psi_i}\oper{J}\oper{A}\ket{\psi_j} - \bra{\psi_i}\oper{A}\oper{J}\ket{\psi_j} = 0\\
		(q_i - q_j)\bra{\psi_i}\oper{A}\ket{\psi_j} = 0\\
		\bra{\psi_i}\oper{A}\ket{\psi_j} = 0 \quad (\text{for } i\not = j)
	\end{gathered}
\end{equation}
So,  if we have a quantum state in a coherent superposition of eigenstates of superselection charge $\oper{J}$ i.e. \(\ket{\psi _+} = \frac{\ket{\psi _1} + \ket{\psi_2}}{\sqrt{2}}\). One can check that:
\begin{equation}
	\begin{gathered}
		\bra{\psi_+}\oper{A}\ket{\psi_+} =  \frac{\bra{\psi_1}\oper{A}\ket{\psi_1} +  \bra{\psi_2}\oper{A}\ket{\psi_2} + \overbrace{2\mathrm{Re}[\bra{\psi_1}\oper{A}\ket{\psi_2}]}
			^{=0}
		}{2}\\
		\bra{\psi_+}\oper{A}\ket{\psi_+} =  \frac{\bra{\psi_1}\oper{A}\ket{\psi_1} +  \bra{\psi_2}\oper{A}\ket{\psi_2}}{2} = \Tr{\rho \oper{A}}
	\end{gathered}
\end{equation}
where
\begin{equation}
	\rho = \frac{\ket{\psi_1}\bra{\psi_1} + \ket{\psi_2}\bra{\psi_2}}{2}
\end{equation}
i.e. any relative phase between \(\ket{\psi_1}\) and \(\ket{\psi_2}\) is not measurable 
and that coherent superposition of \(\ket{\psi_1}\) and \(\ket{\psi_2}\) cannot be 
verified (or prepared). For an observer, the state \(\ket{\psi_+}\) will not be any 
different from a mixed state \(\rho\). 