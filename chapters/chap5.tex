\chapter{Quantum Relational Dynamics of Jaynes-Cummmings Model
\label{chap5:RDQ_JCM_chap}}

In \refchap{chap:braun_briggs_jaynes}, we delved into the semi-classical approach to derive the Time-Dependent Schrödinger Equation (TDSE) governing the Jaynes-Cummings model. 
This method assumes a semiclassical (WKB-like) wave function for the field mode. However, 
this approach inherits the concept of ``time" as a purely classical parameter, thereby compromising its deeper and more fundamental nature.

In contrast, in \ref{chap:sebgem_Rel}, we presented a purely quantum framework to describe the dynamics of the general interacting system. Here, ``time" 
emerges as a symmetry parameter stemming from the principle of global invariance of the global energy 
eigenstate of total Hamiltonian.

This chapter explores the Jaynes-Cummings model from the perspective of Quantum Relational Dynamics. 
We aim to scrutinize the transition from a purely quantum system to semi-classical regimes, 
as previously discussed in \refchap{chap:braun_briggs_jaynes}, and to juxtapose the dynamics as 
characterized by the two distinct approaches.We demonstrate that, under appropriate limits
(consistent with the semi-classical treatment), the two approaches yield identical results. 
Interestingly, the analysis also hints at the presence of a superselection rule.

We assume a special