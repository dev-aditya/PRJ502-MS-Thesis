\chapter{Quantum Relational Dynamics of Jaynes-Cummmings Model
\label{chap5:RDQ_JCM_chap}}

In \refchap{chap:braun_briggs_jaynes}, we delved into the semi-classical approach for 
deriving the Time-Dependent Schrödinger Equation (TDSE) governing the Jaynes-Cummings model. 
This method assumes a semi-classical (WKB-like) wave function for the field mode. However, 
this approach inherits the concept of ``time" as a purely classical parameter, thereby 
compromising its deeper, more fundamental nature.

In contrast, in \ref{chap:sebgem_Rel}, we presented a purely quantum framework for 
describing the dynamics of general interacting system. Here, ``time" 
emerges as a symmetry parameter stemming from the principle of global invariance of the global energy 
eigenstate of total Hamiltonian.

This chapter explores the Jaynes-Cummings model from the perspective of Quantum Relational Dynamics.. 
We aim to scrutinize the transition from a purely quantum system to semi-classical regimes, 
as previously discussed in \refchap{chap:braun_briggs_jaynes}, and to juxtapose the dynamics as 
characterized by the two distinct approaches.We demonstrate that, under appropriate limits 
(consistent with the semiclassical treatment), the two approaches yield identical results. 
Interestingly, the analysis also hints at the presence of a superselection rule.

We assume a special