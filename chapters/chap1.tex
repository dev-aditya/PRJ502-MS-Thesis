\chapter{Introduction\label{chap:introduction}}

\epigraph{ \myopeningquote Time is an illusion \myclosingquote}{-- Albert Einstein}



The unification of Quantum Mechanics and General Relativity has long been
the Holy Grail of physics since their development.
Despite extensive research, no satisfactory theory has yet been
proposed to solve this challenge. General relativity states that a physical
theory should not depend on background structures.
However, conventional quantization methods often introduce background structures,
such as imposing the canonical commutation relations on constant-time hypersurface. The Hamiltonian of a
generally covariant theory, such as general relativity, is constrained to
vanish in the absence of boundaries~\cite{gielen2023quantum}. Attempts to quantize gravity using 
canonical quantization lead to a Hamiltonian constraint, known as the Wheeler-DeWitt equation
(i.e., \(\oper{H} \ketPsi = 0\)). This leads to an infamous problem known as
``The problem of time'' in the canonical approach to quantum gravity
[Refer to \refapp{appen:problemoftime} for more details]. The issue is that
quantum states of space-time (and matter in it) do not seem to undergo any time evolution, as
dictated by the constraints of the theory. 

In quantum cosmology, the universe is described by a wave function whose dynamics is governed by the Wheeler-DeWitt equation, the quantized Hamiltonian constraint of the system.
Due to inherent mathematical ambiguities in the Wheeler-DeWitt equation, it is often unreliable
to extend beyond the semi-classical approximation~\cite{cooke2010qcintro}. The wave function
in this approximation is provided by the WKB wave function, \(\Psi_{\mathrm{WKB}} 
\approx \exp\left[iS_0/\hbar\right]\psi(\{x_n\})\); where \(S_0\) is 
a function which obeys the classical Hamilton-Jacobi equation for the gravitational
field~\cite{gielen2023quantum}. Then we substitute this WKB ansatz into the
Wheeler-DeWitt equation and applying the relevant approximations leads to the time-dependent 
Schr\"odinger equation.  The work of Briggs et.al.  ~\cite{briggsBraun2004, briggs2001derivation} 
have proposed a similar solution to the problem, but from an atomic and molecular
physics perspective, and their approach is discussed in~\refchap{chap:briggs_rost_semiclassic}.

On the other hand, for locally relativistic quantum field theories, it has been demonstrated
that the long-range Coulomb field causes the total ``charge'' operator to commute
with all quasi-local observables, implying a superselection rule for charges~\cite{Strocchi:1974xh} 
[Refer to \refapp{appen:superselection} for further details].
Analogously, due to the coupling of energy with the long-range gravitational field, 
there is speculation about the existence of a superselection rule for energy~\cite{page1983evolution}.
This implies that only operators that commute
with the Hamiltonian can be observable. However, such observables are stationary
(i.e., constants of motion). How can we observe time evolution if the only observables
we can observe are stationary i.e., they do not evolve at all?

The problem of time is a manifestation of background independence rather than the
``timelessness'' of quantum mechanics, and means that physical states do not evolve
relative to an external background time. Instead, time evolution must be extracted relationally, 
by selecting certain quantized degrees of freedom to serve as an internal timekeeper, 
i.e., pick some quantized degrees of freedom to serve as an internal time. 
\texttt{Quantum mechanics typically depicts physical reality as a vector in Hilbert space at a distinct point in time, 
with Schrödinger's equation governing its progression. This equation features an externally imposed time parameter, 
which can be problematic if the universe is viewed as a quantum mechanical system. This is because there shouldn't be 
any external time in such a system.}
Even in our everyday ``classical' world, we never directly measure time; instead, we measure the
position or angular displacement of a pendulum or a dial and use that measurement to \emph{define} a
unit of time. The use of relational time in Quantum
Mechanics is a framework in which one promotes all variables to quantum operators and
later chooses one of the variables to operate like a ``clock'' degree of freedom.

Apart from the semi-classical approaches employed in quantum cosmology, various other 
avenues exist to address the time problem from a purely quantum mechanical standpoint~\cite{hohn2021trinity}. 
One such approach is the ``Page-Wootters Formalism'' (abbreviated P\&W formalism), 
which defines relational dynamics in terms of conditional probabilities for the clock and 
evolving degrees of freedom~\cite{page1983evolution}. To date, no one has successfully tackled 
the problem involving general interaction potentials [with the exception of~\cite{Smith:2017pwx} 
to a certain extent]. The original formulation due to P\&W is no exception, as it also neglects 
the interaction between the environment and the system. However, recent work by Gemsheim and Rost 
has developed a relational formalism based on the P\&W formalism, which now includes the 
environment-system interaction~\cite{Gemsheim:2023izg}. This advancement provides the necessary 
tools to study relational dynamics for a general system-environment setting. Their approach is 
discussed in detail in~\refchap{chap:sebgem_Rel}.


The primary objective of this work is to study the transition from quantum mechanical approach 
to semi-classics, in the limit of a  large environment
\footnote{The notion of a ``large" environment will be elaborated upon
in subsequent chapters.}


\texttt{\cite{Mendes_2019} Has a very nice introduction to the approach, especially the 
Page1 column 2, first paragraph}

\newpage