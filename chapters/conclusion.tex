\chapter{Conclusion and Outlook\label{chap:conclusion}}




Our work investigated the Quantum Relational approach to extract the dynamics within the Jaynes-Cummings Hamiltonian System. We employ both the rotating wave approximation (RWA) and the non-rotating wave approximation (non-RWA) for our analysis, and we successfully identified the conditions necessary to treat the environment as classical within a relational framework.

\refchap{chap:braun_briggs_jaynes} detailed the derivation of the Time-Dependent Schrödinger Equation utilizing semi-classical techniques based on the WKB approximation. It further discussed the coherent state-based semi-classical approach as an alternative method for deriving the TDSE. This approach was then applied to obtain the semi-classical Hamiltonian for the JC-Hamiltonian without the rotating wave approximation.

In \refchap{chap:sebgem_Rel}, we focused on the emergence of time from symmetry arguments applied to the global entangled energy eigenstate. It then elaborated on incorporating generic interacting quantum systems within the P\&W formalism and provided an illustrative example using a simple model. This established a general framework for extending the formalism to arbitrary interactions between subsystems within a global Hilbert space. \refchap{chap5:RDQ_JCM_chap} addressed the JC-Hamiltonian in the RWA context, diagonalizing it for the resonant case. 

Furthermore, \refsec{sec:chap5_large_amplitude_weak_g} of \refchap{chap5:RDQ_JCM_chap} explored the semi-classical limit of the JC-Hamiltonian in the large amplitude and weak interaction regime. We were able to derive the exact analytical form of the unitary time evolution operator under those conditions.\refsec{sec:chap5_numerical_results}  presented the numerical results obtained for the dynamics of the two-level quantum subsystem using both the relational and semi-classical approaches. These results were compared for both RWA and non-RWA cases. As anticipated and theoretically derived in previous chapters, our findings demonstrate that the dynamics described by the Quantum Relational theory converge to those predicted by semi-classical approaches when the environment can be treated classically.

In addition to providing a fundamental interpretation of ``time" based on symmetry arguments applied to the global energy eigenstate and the entanglement within the subsystems, relational formalism presents a powerful framework for obtaining exact analytical results for subsystem dynamics in the presence of complex potentials. This holds true if the conditional clock states are chosen in a way that generates the desired emergent potential, as described by \refeq{eq:chap3_effective_Vs}. This has significant implications for both analytical and numerical methods in physics.

However, several open questions remain in our analysis, demanding further investigation. 
\begin{itemize}
    \item Thus far, we have only considered dividing the global Hilbert space into two subspaces, one representing the environment and the other the system of interest; future work could explore the definition of the semi-classical limit for scenarios involving more than two subspaces. 
    \item Additionally, a critical question remains: what types of environmental states within a generic quantum system would serve as the best candidates for the conditional clock states, particularly in the context of semi-classical theories? 
    \item  This framework highlights the crucial role of entanglement in extracting dynamics and underpins the emergence of time. However, the precise dependence of our findings on the nature of dynamics and entanglement remains an open question.
\end{itemize}
