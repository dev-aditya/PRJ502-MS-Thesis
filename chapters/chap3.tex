\chapter[Quantum Relational Dynamics]{Relational Quantum Mechanics: With System and Environment Interaction\label{chap:sebgem_Rel}}

[\textit{Please note that the words `clock' and `environment' will be used synonymously. The term ``system" denotes a subsystem (excluding the environment) of a global Hilbert space unless otherwise specified.}]

We start by \emph{reformulating} the time independent Schr\"odinger equation as an \emph{invariance principle}~\cite{Gemsheim:2023izg}
\begin{equation}
    \label{eq:invariance_principle}
        \exp\left[i\lambda(\H - E\I)\right] \ketPsi = \ketPsi. 
\end{equation}
Where  \(\lambda\) is any complex-valued parameter, $\H$ is the Hamiltonian of the global system, and \(\ketPsi\) is the global state vector. Differentiating (\refeq{eq:invariance_principle}) with respect to \(\lambda\) one gets  the usual Time Independent Schr\"odinger Equation (TISE)
\begin{equation*}
    \H \ketPsi = E \ketPsi.
\end{equation*}

For our analysis, we assume that λ is real. We partition the total Hilbert space $\mathcal{H}$ into two components to extract the dynamics. A system \(\mathcal{H}_s\) and a clock \(\mathcal{H}_c\). We re-write 
\begin{equation}
    \H = \H_s \otimes \I_c + \I_s \otimes \H_c + \Vinter .
\end{equation}

Since the system and environment are embedded in the global Hilbert space, one can single
out the system state by projecting the global state partially onto a static state of the environment. 

\paragraph{For non-interacting system and environment:} Let \(\ket{\chi}\) be a state vector 
in clock Hilbert space. 

We choose some state \(\ket{\chi_0} \in \mathcal{H_c}\). If one projects the state vector \(\ket{\chi_0}\) onto the invariance equation~(\refeq{eq:invariance_principle}) (assuming the absence of interaction, represented by \(\Vinter = \oper{0})\), one gets
\begin{equation}
\begin{gathered}
\bra{\chi_0}e^{i\lambda(\H - E)}\ketPsi = \innerGlob{\chi_0}{\Psi}\\
\bra{\chi_0}e^{i\lambda(\H_c - E)}\ketPsi =e^{-i\lambda \H_s}  \innerGlob{\chi_0}{\Psi}.
\end{gathered}
\end{equation}
We define
\begin{equation}
   \ket{\chi_\lambda} = e^{-i\lambda(\H_c - E)}\ket{\chi_0}.
\end{equation}

By conditioning the global state \(\ketPsi\) onto a clock state \(\ket{\chi_\lambda}\),
we utilize the state \(\ket{\chi_\lambda}\) as a label to associate the system state with the particular value of \(\lambda\), i.e., 
\begin{equation}
    \ket{\varphi(\lambda)}_s \equiv \innerGlob{\chi_\lambda}{\Psi}.
\end{equation}
So, 
\begin{equation}
\label{eq:nonint_sys_evol}
    \ket{\varphi(\lambda)}_s = e^{-i\lambda \H_s} \innerGlob{\chi_0}{\Psi} \equiv e^{-i\lambda \H_s}\ket{\varphi(0)}_s.
\end{equation}

Notice that the choice of \(\ket{\chi_0}\) fixes the system's initial state.
Since, \(\lambda\) in (\refeq{eq:nonint_sys_evol}) is assumed to be a continuous 
 parameter, the above equation can be interpreted as a solution to the differential equation
\begin{equation}
    i \dfrac{d}{d\lambda} \ket{\varphi(\lambda)}_s = \H_s \ket{\varphi(\lambda)}_s.
\end{equation}
Which is equivalent to (Time-dependent Schr\"odinger Equation) TDSE in units 
\(\hbar = 1\). It is crucial for the system and the environment within the global Hilbert space to exhibit entanglement. Otherwise, the system and the environment would uphold separate ``global" invariance principles, each with its own parameter, \(\lambda_s\) and \(\lambda_c\), respectively. This would leave the relationship between \(\lambda_s\) and \(\lambda_c\) unresolved~\cite{Gemsheim:2023izg}.

\paragraph{Role of parameter \(\lambda\):} The variable \(\lambda\) introduced 
in the above derivation has no physical significance. It only serves as a parameter
to track the evolution of our system. Any reparametrization of \(\lambda \to t(\lambda)\) 
has no change in the equation of the system state evolution. One can, in principle 
\footnote{As one does it always. We never measure time directly; rather, we measure
the angular position of a clock dial or the no of transition electrons made in a cesium atom.}
 parameterize the evolution using an observable of our environment 
 \(\operatorname{A}_c(\lambda)\equiv \bra{\chi_\lambda}\oper{A}_c\ket{\chi_\lambda}\). 
 An ideal choice would be to use such a $\oper{A}_c$ for which the relation between $\lambda $ and $\operatorname{A}_c$ is simple, for example, linear.

High-resolution clock states $\ket{\chi_\lambda} \propto \sum_k a_k \exp (-i\lambda E_c ^k)\ket{E_c}_k$ require a broad distribution over the clock energy eigenstates, ideally with uniform coefficients ($a_k$)\\\cite{Gemsheim:2023izg, Smith:2017pwx}. This condition is readily met when the clock's physical dimensions exceed those of the system.


\paragraph{For interacting system and environment:} So far in our analysis, we have assumed 
no interaction, i.e., \(\Vinter = \oper{0}\). However, in real scenarios, one always has some
interaction within components of a global closed system. To extend the derivation to 
non-zero \(\Vinter\), we modify our clock state as
\begin{equation}
   \begin{gathered}
      \ket{\chi_\lambda} = e^{-i\lambda(\H_c - E)}\ket{\chi_0} \to \\\ket{\chi_\lambda}
       = e^{-iS(\lambda)}e^{-i\lambda(\H_c - E)}\ket{\chi_0}
   \end{gathered}
\end{equation}
where \(S(\lambda) = \int^{\lambda}\xi(\tilde{\lambda} ) d\tilde{\lambda}\) is an extra factor introduced for 
simplifying upcoming derivations. When  projected onto this, the global state can be written as
\begin{equation}
\label{eq:TISE_undecomposed}
    \left(-\H_s + \xi(\lambda) + i\dfrac{d}{d\lambda}\right) 
    \innerGlob{\chi_\lambda}{\Psi} = \bra{\chi_\lambda}\Vinter\ketPsi.
\end{equation}
We can rearrange the above equation to write it as
\begin{equation}
\label{eq:TISE_undecomposed_simp}
    \begin{gathered}
         i\dfrac{d}{d\lambda}\innerGlob{\chi_\lambda}{\Psi} =
          \H_s\innerGlob{\chi_\lambda}{\Psi} - \xi(\lambda)\innerGlob{\chi_\lambda}{\Psi}  + \bra{\chi_\lambda}\Vinter\ketPsi.
    \end{gathered}
\end{equation}

We now decompose $\bra{\chi_\lambda}\Vinter\ketPsi$ into a Hermitian potential and a c-number. To facilitate the decomposition, we define the following:
 \begin{equation}
     \begin{gathered}
         \oper{P}_\Psi =\kket{\Psi} \bbra{\Psi}, \quad \oper{P}_\chi = \I_s \otimes \ket{\chi_\lambda}\bra{\chi_\lambda}\\
         \oper{P}_{\Psi \chi} = \oper{P}_\Psi\oper{P}_\chi /\mathcal{N}_\lambda
     \end{gathered}
 \end{equation}
where \(\mathcal{N}_\lambda = \bbra{\Psi}\oper{P}_\chi\ketPsi\). One observes that
\begin{equation}
\oper{P}_{\Psi \chi}\ketPsi = \frac{\oper{P}_\Psi\oper{P}_\chi}{\mathcal{N}_\lambda} \ketPsi
    = \ketPsi
\end{equation}
 Using above defined operators, we decompose the 
 \begin{equation}
 \label{eq:effective_Vs}
     \begin{aligned}
         \bra{\chi_\lambda}\Vinter\ketPsi = \bra{\chi_\lambda}\Vinter\oper{P}_{\Psi \chi}\ketPsi\\
         = \left[\oper{V}_s (\lambda)- \bbra{\Psi} \oper{V}\oper{P}_\chi \ketPsi/\mathcal{N}_\lambda \right] \innerGlob{\chi_\lambda}{\Psi}
     \end{aligned}
 \end{equation}
where 
\begin{equation*}
    \Vs(\lambda) = \frac{\bra{\chi_\lambda} \left(\Vinter\oper{P}_{\Psi} + \oper{P}_{\Psi}\Vinter\right)\ket{\chi_\lambda}}{\mathcal{N}_\lambda}
\end{equation*}

Inserting (\refeq{eq:effective_Vs}) into (\refeq{eq:TISE_undecomposed_simp}) and setting \(\xi(\lambda) = -\bbra{\Psi} \oper{V}\oper{P}_\chi \ketPsi/\mathcal{N}_\lambda\) we obtain
\begin{equation}
\label{eq:chap3_sebgem_TDSE}
 \boxed{ i\dfrac{d}{d\lambda}\ket{\varphi(\lambda)}_s= \left[\H_s + \Vs (\lambda)\right]\ket{\varphi(\lambda)}_s},
\end{equation}
where we defined \(\innerGlob{\chi_\lambda}{\Psi} = \ket{\varphi(\lambda)}_s\).  The phase factor  \(S(\lambda) = \int^\lambda \xi(\tau)d\tau \) introduced helps eliminate the c-number after the decomposition of the ``effective'' potential. 

We find that the final form of the Equation is equivalent to the TDSE for the remaining sub-system (\(\mathcal{H}_s\)) of the
global Hilbert space \(\mathcal{H}\).

\newpage