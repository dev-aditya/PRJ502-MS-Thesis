\chapter{Relational Dynamics of Jaynes Cumming Model
\label{chap:braun_briggs_jaynes}}

It has been shown in \refchap{chap:briggs_rost_semiclassic} how one can reduce a TISE 
to a TDSE for a system in the presence of a classical environment. In this chapter, 
we will discuss the application of the previous chapter to a simple model of light-matter 
interaction and enumerating the conditions that must be obtained in order to treat the 
electromagnetic field classically. The upcoming sections outlines the derivations as presented 
in~\cite{braun2004classical}

We start by considerig a time-independent hamiltonian of an interacting quantum system,
with a Hamiltonian given as
\begin{equation}
    H = H_s(p, x) + H_F(P, Q) + H_I(x, Q). 
\end{equation}
where $(x, p)$ and $(Q, P)$ are (position, momentum) operator for system and boson field, respectively. 
Also, $H_s$ is the Hamiltonian of the atomic system, $H_F$ is the field hamiltonian, and $H_I$ is the interaction hamiltonian. 
The boson field hamiltonian will be takes a sum over field modes, with hamiltonian given as 
\begin{equation}
    H_F(P, Q) = \sum_{k} \frac{1}{2}\left(P_k ^2 + \omega_k^2 Q_k^2\right)
\end{equation}

We write the above Hamiltonian in terms of creating and annihilation operators
\begin{equation}
    \label{eq:class_jcm_eq1}
    H = \sum_i \epsilon_i c^\dagger _i c _i + \sum_k \hbar \omega_k \left(a^\dagger _k a_k 
    + \frac{1}{2}\right) + \hbar\sum_k  g_{jk}^k c^\dagger _i c _i\left(a^\dagger _k + a_k \right).
\end{equation}
where $c_i$ and $c^\dagger _i$ are the annihilation and creation atomic operators for the system, 
and $a_k$ and $a^\dagger _k$ are the annihilation and creation operators for the boson field.

A sepcial case of the above Hamiltonian is when we take a two level atom (or a spin 1/2 system)
interacting with a single mode of the electromagnetic field. \refeq{eq:class_jcm_eq1} then reduces to
\begin{eqnarray}
    \label{eq:class_jcm_eq2}
    H = \hbar \omega_0 \sigma_z + \hbar \omega \left(a^\dagger a + \frac{1}{2}\right) 
    + \hbar g \sigma_x\left(a + a^\dagger\right).
\end{eqnarray}

\section{Semiclassical limit of the JC-Model\label{sec:class_jcm_sec1}}
As done in \refeq{eqn:chap2_total_wavefunction}, one can write in general the solution of 
TISE 
\begin{equation}
    \label{eq:class_jcm_eq2}
    (H - E)\Psi = 0
\end{equation}
with $H$ in the form of \refeq{eq:class_jcm_eq1} as 
\begin{equation}
    \label{eq:class_jcm_eq3}
    \Psi(x, Q) = \sum_{k} \chi_k(Q) \phi_k(x, Q).
\end{equation}

We employ this particular form, wherein $\chi_k(Q)$ displays no reliance on $x$, 
under the assumption that the system exerts minimal back-reaction on the field.
This assumption has been utilized previously in \refchap{chap:briggs_rost_semiclassic} 
and will be employed again in forthcoming derivations.

Doing the same exercise as in \refeq{eqn:briggs_rost_eq5}, we substitute
\refeq{eq:class_jcm_eq3} into \refeq{eq:class_jcm_eq2} and obtain a set of coupled equations

\begin{align}
    \label{eq:class_jcm_eq4}
   \sum_i \chi_i(Q) \left[H_s + H_I - \left(E - \sum_k
    \frac{1}{2}\omega_k^2 Q_k^2 + \frac{\hbar^2}{2\chi_i}\dfrac{\partial^2}{\partial Q_k^2}
    \chi_i\right) \right]\nonumber \\
    + \sum_i \chi_i(Q) \left[
        \frac{-\hbar^2}{2}\dfrac{\partial^2}{\partial Q_k^2} - 
        \frac{-\hbar^2}{\chi_i}\dfrac{\partial}{\partial Q_k}\chi_i\dfrac{\partial}{\partial Q_k}
    \right]\phi_i (x, Q) = 0.
\end{align}

\refeq{eq:class_jcm_eq4} when projected onto state \(\phi_j\) gives
\begin{align}
    \label{eq:class_jcm_eq5}
    \sum_k \left[-\frac{\hbar^2}{2}\dfrac{\partial^2}{\partial Q_k^2}
     + \frac{1}{2} \omega_k^2 Q_k^2 \right]\chi_j
    + \sum_i \bra{\phi_j}H_s + H_I\ket{\phi_i}\chi_i \nonumber \\
   - \sum_{i, k} \left[
    \bra{\phi_j}\frac{\hbar^2 \partial^2}{2 \partial Q_k^2}\ket{\phi_i}
    + \bra{\phi_j}{\hbar^2}\dfrac{\partial}{\partial Q_k}\ket{\phi_i}\dfrac{\partial}{\partial Q_k}
   \right]\chi_i = E \chi_j.
\end{align}

The equation above describes the ``close-coupled" form for \(\chi_j\). 
The off-diagonal terms account for changes in the state of the boson
field caused by changes in the system's state. Neglecting these 
off-diagonal coupling terms simplifies the equation, providing 
the state of \(\chi_j\) when the system is in state \(\phi_j\), i.e., 
\begin{equation}
    \begin{aligned}
        \label{eq:class_jcm_eq6}
    \left[\sum_k \left(-\frac{\hbar^2}{2}\dfrac{\partial^2}{\partial Q_k^2}
    + \frac{1}{2} \omega_k^2 Q_k^2\right)
    + E_j(Q) - E \right]\chi_j \\
    = \hbar^2 \sum_k \bra*{\phi_j}H_s + H_I \ket*{\phi_j}\dfrac{\partial}{\partial Q_k}\chi_j.
    \end{aligned}
\end{equation}
with 
\begin{eqnarray}
    E_j(Q) = \bra*{\phi_j}H_s + H_I - \sum_k \frac{\hbar^2}{2} \dfrac{\partial^2}{\partial Q_k^2} \ket*{\phi_i}.
\end{eqnarray}
The diagonal \(\bra*{\phi_j}H_s + H_I \ket*{\phi_j}\) terms on the RHS of \refeq{eq:class_jcm_eq6}
are zero for real \(\phi_j\) or else can be eliminated by a (Berry) phase transformation~\cite{braun2004classical}.
In addition, since we are neglecting the first order derivatives w.r.t. $Q_k$, it's consistent to neglect
the second order derivatives as well, which gives us the following equation
\begin{equation}
    \label{eq:class_jcm_eq7}
    \left[\sum_k \left(-\frac{\hbar^2}{2}\dfrac{\partial^2}{\partial Q_k^2}
    + \frac{1}{2} \omega_k^2 Q_k^2\right)
    + E_j(Q) - E \right]\chi_j (Q)= 0.
\end{equation}
this is the defining equation for the state of the boson field when the system is in state \(\phi_j\).

For, complete independence of the field from the system, we need to replace 
\(E_j(Q)\) by some fixed average energy \(\bar{E}(Q)\) and corresponding 
\(\chi_j(Q)\) by some ``mean''-field state \({\chi}(Q)\). Hence, we replace 
\begin{equation}
    \label{eq:class_jcm_eq8}
    \chi_j(Q) = a_j(Q) \chi(Q).
\end{equation}
where we assume \(a_j(Q)\) to be slowly varying function of \(Q\). The \refeq{eq:class_jcm_eq3}
reduces to
\begin{align}
    \label{eq:class_jcm_eq9}
    \Psi(x, Q) = \chi(Q) \sum_j a_j(Q) \phi_j(x) \nonumber \\
    \Psi(x, Q) =  \chi(Q)  \psi(x, Q).
\end{align}
We've identified the key approximations needed to express the exact wave function 
(\refeq{eq:class_jcm_eq3}) in the simpler, factorized form (\refeq{eq:class_jcm_eq9}). 
This step is critical because it assumes the influence of the atom on the field is minimal, while the 
field has a strong influence on the atom. Now, we can focus on deriving the effective 
Schrödinger equation for the wave function (denoted by \(\psi\)) representing the quantum system.

One can in general, view \refeq{eq:class_jcm_eq9} as a general ansatz for the wave function
and find from \refeq{eq:class_jcm_eq2}\footnote{Here we assumed only single mode of E.M field} 
\begin{align}
    \label{eq:class_jcm_eqA}
    (H - E) \Psi = \chi(Q)\left[H_S+H_I(x, Q)-\frac{\hbar^2}{2}\left(\frac{\partial^2}{\partial Q^2}\right.\right.
    & \left.\left. 
    +2 \frac{\chi^{\prime}(Q)}{\chi(Q)} \frac{\partial}{\partial Q}\right)\right] 
    \psi(x, Q) \nonumber \\
    +\psi(x, Q)\left[-\frac{\hbar^2}{2} \frac{\partial^2}{\partial Q^2}+\frac{\omega^2}{2} Q^2
    - E\right] \chi(Q) = 0
\end{align}
To streamline our analysis, we'll now focus on a single field mode. This derivation takes a 
slightly different approach compared to both the method outlined in 
\refchap{chap:briggs_rost_semiclassic} 
and the one we'll use later for the Jaynes-Cummings model (our toy model universe). 
We present this alternative approach to lay the groundwork for the new considerations 
in the next section, which will leverage coherent states.

We split the action of the total Hamiltonian on the field ad quantum system such that the 
wave function \(\chi(Q)\) described the field with energy close to total energy \(E\), while 
the reaming part of the equation describes the quantum system (with negligible energy in camparison).
Completely ignoring the influence of the quantum system on the field (back coupling) is analogous 
to neglecting the terms \(\bar{E}(Q)\) in the equations. This simplification allows us 
to select a wave function for the field, which will ultimately be treated classically. 
This is equivalet to neglecting \(\bar{E}(Q)\) and allows one to choose the wave function of the
field which is to become classical, This wave function must be an eigenstate of the fixed-field Schrödinger equation
, i.e.
\begin{equation}
    \label{eq:class_jcm_eq10}
    \left[\sum_k \left(-\frac{\hbar^2}{2}\dfrac{\partial^2}{\partial Q_k^2}
    + \frac{1}{2} \omega_k^2 Q_k^2\right)
    - E \right]\chi(Q) = 0.
\end{equation}
Under the assumption of these large (classical) energies and assuming we're sufficiently far from any 
classical turning points, we can approximate the actual wave function, denoted by $\chi(Q)$,
 by its WKB approximation, i.e., 
 \begin{equation}
    \label{eq:class_jcm_eq11}
    \chi(Q) = \exp\left(\frac{i}{\hbar}\int ^Q dQ' P(Q')\right).
 \end{equation}
where \(P(Q)\) is the momentum of the field, given as 
\begin{equation}
    \label{eq:class_jcm_eq12}
    P(Q) = \sqrt{2\left(E - \frac{1}{2}\omega^2 Q^2\right)}.
\end{equation}
Using \refeq{eq:class_jcm_eqA} - \refeq{eq:class_jcm_eq11} gives
\begin{eqnarray}
    \label{eq:class_jcm_eq13}
    \left[\sum_k \left(-\frac{\hbar^2}{2}\dfrac{\partial^2}{\partial Q_k^2}
    + \frac{1}{2} \omega_k^2 Q_k^2\right)
    - E \right]\exp\left(\frac{i}{\hbar}\int ^Q dQ' P(Q')\right) = 0 \\
    \left[
        H_s + H_I(x, Q) - \frac{\hbar^2}{2} \dfrac{\partial^2}{\partial Q^2}
    - i\hbar P(Q) \dfrac{\partial}{\partial Q}\right] \psi(x, Q) = 0. 
\end{eqnarray}

The above differential equation describes the dynamics of (remaining part) 
the quantum system. We simplfy this equation by replacing the equation by a parameter $t$, 
which parametrizes $Q(t)$ trajectory and define
\begin{equation}
    \label{eq:class_jcm_eq14}
    P(Q)\dfrac{\partial}{\partial Q} \equiv \dfrac{\partial }{\partial t}
\end{equation}
It's evident that $P(Q)$ derived from \refeq{eq:class_jcm_eq12} corresponds to the velocity 
$\dot{Q}$, as dictated by classical equations of motion. These equations state that
$\dot{Q} = P$ and $\ddot{P} = -\omega^2 Q$, where $Q(t) = Q_0 \cos(\omega t)$ represents 
\textit{the solution, demonstrating that the parameter $t$ signifies classical time.}

We define \(\psi(x, t) = \psi(x, Q(t))\) and find from \refeq{eq:class_jcm_eq13}
\begin{equation}
    \label{eq:class_jcm_eq15}
    i \frac{\partial}{\partial t}\psi(x, t) = \left[ H_s + H_I(x,Q(t)) 
    \frac{\hbar^2}{2} \left(\frac{\ddot{Q}(t)}{\dot{Q}^3(t)}\frac{\partial}{\partial t}
    - \frac{1}{\dot{Q}^2(t)} 
    \frac{\partial^2}{\partial t^2}\right)\right]\psi(x, t).
\end{equation}

The derivatives appearing on the right-hand side stem from the second-order derivative 
$\frac{\partial^2}{\partial Q^2}$ wrt position, when expressed in terms 
of time derivatives. In the scenario of a substantial (``classical") amount of 
energy $E$, predominantly residing in the classical degree of freedom $Q$, these additional 
terms become negligible, as we will soon illustrate. Initially, it's worth noting that 
$\frac{\partial}{\partial t}$ is of the order of energy of the quantum system, 
$E_s = \langle H_s + H_I\rangle$, implying it is considerably smaller than $E$. This allows us to 
disregard the supplementary derivative terms on the right side of \refeq{eq:class_jcm_eq15}. 
Proceeding with self-consistency, and leveraging ${\dot{Q}} = P \approx \sqrt{E}$ 
for the harmonic oscillator away from its turning points, as well as 
$\ddot{Q} \approx \omega \dot{Q}$, we can estimate their respective orders of magnitude.
First, 
\begin{equation}
    \label{eq:class_jcm_eq16}
    \left \langle \frac{\hbar^2}{2} \frac{\ddot{Q}(t)}{\dot{Q}^3(t)}\frac{\partial}{\partial t} \right\rangle
    \approx E_s \frac{\hbar \omega}{E}, \text{ and } \left \langle\frac{\hbar^2}{2} \frac{1}{\dot{Q}^2(t)} 
    \frac{\partial^2}{\partial t^2} \right\rangle \approx \frac{E_s^2}{E^2}.
\end{equation}
The additional derivative terms on the right-hand side of \refeq{eq:class_jcm_eq15} 
exhibit an order of magnitude of $E_s\left(\frac{E_s + \hbar \omega}{E}\right)$. 
Hence, in comparison to the remaining terms on the right-hand side, which are of 
the order $E_s$, these additional terms are diminished by a factor approximately 
$\frac{\hbar \omega}{E} \approx \frac{1}{n}$, where $n$ represents the number of
photons in the field mode.
This factor is significantly small for a classical field. Hence, we can neglet these 
terms and write
\begin{equation}
    \label{eq:class_jcm_eq17}
    i \frac{\partial}{\partial t}\psi(x, t) = \left[ H_s + H_I(x,Q(t))\right]\psi(x, t)
\end{equation}
which is the usual form of TDSE for the quantum system interacting with the classical field.

The derivation relies on the Time-Independent Schrödinger Equation (TISE) for both the 
atom and the field. Time arises as a consequence of classical motion, serving solely as 
a derived classical parameter. It's important to note that 
the aforementioned arguments hold true only away from turning points where 
the velocity ${\dot{Q}}(t)$ is non-zero. But it's evident that the Time-Dependent 
Schrödinger Equation (TDSE) of (\refeq{eq:class_jcm_eq17}) remains valid for all times. 
The reason behind this limitation is apparent: it lies in the selection of $Q$, 
the position representation, for the field mode. 
While this choice aids in diagonalizing the coupling Hamiltonian $H_I$, 
in this representation, the real field quadrature $Q(t)$ undergoes 
harmonic motion with periodic zeros in its time derivative $Q^{\dot{}}(t)$. 
Consequently, the position representation fails to offer a global notion of time. 
Next section will illustrates how this limitation can be addressed by utilizing 
a coherent state representation of the field state.

\section{Coherent State Derivation of TDSE}
In quantum optics, coherent states hold a special place. Representing minimal uncertainty 
in both position $(Q)$ and momentum $(P)$ of the field mode, these states exhibit classical 
behavior. It's therefore unsurprising that coherent states play a crucial role in deriving 
a time-dependent Schrödinger equation for the quantum system's degrees of freedom. This 
derivation leverages a time-independent Schrödinger equation encompassing both the coupled 
quantum system and the field mode. 

We begin by writing  the total hamiltonian for single mode of E.M field in term of usual 
creation and annihilation operators as
\begin{equation}
    \label{eq:class_jcm_eq18}
    \begin{aligned}
        \H_F = \hbar \omega \left(\oper{a}^\dagger \oper{a} + 1/2\right), \\
        \H_I = \hbar \oper{S} (\oper{a} + \oper{a}^\dagger)
    \end{aligned}
\end{equation}
with \(\oper{S} = \sum_{ij}^k g_{ij} \oper{c}_i^\dagger \oper{c}_i\) as in \refeq{eq:class_jcm_eq1}.

Coherent state, defined as, \(\ket{\alpha} = \exp\left(\abs*{\alpha}^2/2 + \alpha \oper{a}^\dagger\right) \ket{0}\)
where \(\alpha\) is a complex number, is an eigenstate of the annihilation operator, i.e., 
\(\oper{a}\ket{\alpha} = \alpha \ket{\alpha}\). Some crucial properties that will be of use to us are 
\begin{equation}
    \label{eq:class_jcm_eq19}
    \begin{aligned}
        \bra*{\alpha}\oper{a}^\dagger = \alpha^* \bra*{\alpha},\\
        \bra*{\alpha}\oper{a} = \left(\dfrac{\partial}{\partial \alpha^*} + \frac{\alpha}{2}\right)\bra*{\alpha}.
    \end{aligned}
\end{equation}

Coherent states form an overcomplete basis, i.e., any state \(\ket{\psi}\) can be expanded as
\begin{equation}
    \label{eq:class_jcm_eq20}
    \ket{\psi} = \int \frac{d^2\alpha}{\pi} \ket{\alpha}\bra*{\alpha}\ket{\psi} \equiv 
    \int \frac{d^2\alpha}{\pi} \ket{\alpha}\Tilde{\chi}(\alpha, \alpha^*).
\end{equation}

where \(\Tilde{\chi}(\alpha, \alpha^*) = \bra*{\alpha}\ket{\psi}\). Note that we can 
always write  
\[\Tilde{\chi}(\alpha, \alpha^*) = \chi(\alpha^*)\exp\left(-\abs*{\alpha}/2\right)\]
where \(\chi(\alpha, \alpha^*)\) is a complex function of \(\alpha\). Instead of utilizing the position
$Q$ representation as previously done, we leverage the properties of coherent states mentioned 
above to express the Time-Independent Schrödinger Equation (TISE) within the coherent state 
representation relative to the classical degree of freedom. We write the total wave function
\begin{equation}
    \label{eq:class_jcm_eq21}
    \braket*{\alpha}{\Psi} = \exp\left(-\abs*{\alpha}^2/2\right)\chi(\alpha^*)\psi(\alpha^*).
\end{equation}
As before, we will consider \( \exp\left(-\abs*{\alpha}^2/2\right)\chi(\alpha^*)\) to 
describe the classical degree of freedom, while \(\psi(\alpha^*)\) will represent the 
quantum system without backreaction to the classical degree of freedom.

Another important property of the conherent state is it's overlap with the  photon number state, 
which is given as \( \braket*{n}{\alpha} = \exp\left(-\abs*{\alpha}/2\right) (\alpha^*)^n/\sqrt{n!} \). 
In the limit of large photon number \(n\), it's easy to see that only those coherent states 
\(\ket{\alpha}\) with 
\begin{equation}
    \label{eq:class_jcm_eq22}
    \abs*{\alpha}^2 = n
\end{equation}
will contribute significantly to the number state \(\ket{n}\).


Another significant property of coherent states is their overlap with photon number states, expressed as 
\( \braket*{n}{\alpha} = \exp\left(-\abs*{\alpha}/2\right) (\alpha^*)^n/\sqrt{n!} \). 
When considering a large photon number $n$, it becomes evident that only coherent states 
$\ket{\alpha}$ satisfying the condition:
\begin{equation}
\label{eq:class_jcm_eq22}
\abs*{\alpha}^2 = n
\end{equation}
contribute significantly to the corresponding number state \(\ket{n}\). 
With \refeq{eq:class_jcm_eq21} in mind, the TISE with $H_F$ and $H_I$ from 
\refeq{eq:class_jcm_eq20} now reads
\begin{eqnarray}
    \label{eq:class_jcm_eq23}
    \bra*{\alpha}(H - E)\ket*{\Psi} = \chi(\alpha^*)
    \left[H_s + \hbar g \left(\alpha ^* + 
    \frac{\chi'(\alpha^*)}{\chi(\alpha^*)} + \dfrac{\partial}{\partial \alpha^*}
    \right)\sigma_x + \hbar \omega \alpha^* \dfrac{\partial}{\partial \alpha^*}\right] 
    \ket*{\psi(\alpha^*)} \nonumber \\
    + \ket*{\psi(\alpha^*)}
    \left[\hbar \omega \left(
        \alpha^* \dfrac{\partial}{\partial \alpha^*} + \frac{1}{2}\right) - E\right]\chi(\alpha^*) = 0. \quad \qquad
\end{eqnarray}

The above equation corresponds to \refeq{eq:class_jcm_eqA} of position space approach. 
Just as we did in \refsec{sec:class_jcm_sec1}, we chose \(\chi\) to be an eigenstate of the field 
Hamiltonian, i.e., to be a number state \(\ket*{n}\) with energy \(E = \hbar \omega \left[n + 1/2\right]\).
In coherent state representation, 
\begin{eqnarray}
    \label{eq:class_jcm_eq24}
    \chi(\alpha^*) =  (\alpha^*)^n/\sqrt{n!}. 
\end{eqnarray}

This ensures that the second part of \refeq{eq:class_jcm_eq23} dissappears and one notices that
\begin{eqnarray}
    \label{eq:class_jcm_eq25}   
    \frac{\chi'(\alpha^*)}{\chi(\alpha^*)} = \frac{n}{\alpha^*} = \frac{\alpha^*\alpha}{\alpha^*} = \alpha.
\end{eqnarray}

Using \refeq{eq:class_jcm_eq24} and \refeq{eq:class_jcm_eq25} in \refeq{eq:class_jcm_eq23} gives
\begin{equation}
    \label{eq:class_jcm_eq26}
    \left[H_s + \hbar S \left(\alpha^* + \alpha + 
    \dfrac{\partial}{\partial  \alpha^*}\right) 
    + \hbar \omega \alpha^* \dfrac{\partial}{\partial \alpha^*} \right] \ket*{\psi(\alpha^*)} = 0.
\end{equation}
We now introduce a new parameter \(t\), which replaces \(\alpha\) analogous to what we did 
in position space representation. The parameter \(t\) is defined through a complex trajectory 
\(\alpha^*(t)\) for the coherent state field amplitude, such that 
\begin{equation}
    \label{eq:class_jcm_eq27}
    \hbar \omega \alpha^* \dfrac{\partial}{\partial \alpha^*} \equiv 
    -i \hbar\dfrac{\partial}{\partial t}.
 \end{equation}
Hence, time is determined by the classical motion of the field amplitude, which in this case is 
\begin{equation}
    \label{eq:class_jcm_eq28}
    \alpha(t) = \alpha_0 e^{-i\omega t}.
\end{equation}

Importantly, whereas the position space expression (\refeq{eq:class_jcm_eq14}) becomes 
problematic near classical turning 
points due to $P(Q) = 0$, the coherent state expression (\refeq{eq:class_jcm_eq27}) maintains finiteness throughout all times. 
Furthermore, it's noteworthy that the coherent state equation (\refeq{eq:class_jcm_eq26}) exclusively involves first-order 
derivatives. This characteristic closely resembles the first-order time-dependent Schrödinger 
equation, whereas the position space expression (\refeq{eq:class_jcm_eq15}) is of second order.

We see that the field part of the interaction Hamiltonian becomes, 
\begin{eqnarray}
    \label{eq:class_jcm_eq29}
    \left[\alpha^* + \alpha - \alpha \left(\hbar \omega \abs*{\alpha}^2\right)^{-1}
    \left(i \hbar \partial / \partial t\right) \right]
\end{eqnarray}

In large photon number limit \(1/\abs*{\alpha}^2 \approx 1/n\), vanishes. This can be 
compared to the similar discussion after \refeq{eq:class_jcm_eq15}. Therefore we find that, in 
the limit of large photon number, the TDSE for state \(\ket*{\psi(t)}\) is 
\begin{mdframed}
    \begin{equation}
        \label{eq:class_jcm_eq30}
        i\hbar \dfrac{\partial}{\partial t} \ket*{\psi(t)} = \left[
            H_s +  \hbar S \left(\alpha_0 e^{-i \omega t} + \alpha_0 e^{i \omega t}\right) 
        \right] \ket*{\psi(t)}.
    \end{equation}
\end{mdframed}

Under the assumption of a valid two-level approximation, the system can be represented 
by the operator $S = g\sigma_x$.  Crucially, unlike the equivalent result obtained in 
position space representation, the derivation of \refeq{eq:class_jcm_eq30} using coherent 
states is valid for all times.

Furthermore, this equation for a two-level system interacting with a single electromagnetic 
field mode is of critical importance. We will leverage it to compare the system's dynamics 
obtained through this approach to those obtained from a fully quantum relational dynamics 
approach in the next section.